
\documentclass[11pt]{article}

\usepackage[latin1]{inputenc}
\usepackage{amssymb}
\usepackage{amsmath}
\usepackage{amscd}
\usepackage{amsthm}
\usepackage{amsfonts}
\usepackage{enumerate}
\usepackage{graphicx}
\usepackage{url}
\usepackage[breaklinks=true,hyperref]{hyperref}
\usepackage{amssymb}
\usepackage[dvips]{color}
\usepackage{epsfig}
\usepackage{mathrsfs}
\usepackage{indentfirst}
\usepackage{subfig}


\pdfpagewidth 8.5in
\pdfpageheight 11in
\topmargin -1in
\headheight 0in
\headsep 0in
\textheight 8.5in
\textwidth 6.5in
\oddsidemargin 0in
\evensidemargin 0in
\headheight 75pt
\headsep 0in
\footskip .75in


\newenvironment{ee}{\begin{enumerate}}{\end{enumerate}}
\newenvironment{ii}{\begin{itemize}}{\end{itemize}}


\newcommand{\argmax}{\arg\!\max}
\newcommand{\argmin}{\arg\!\min}

\newcommand{\Var}{\text{Var}}
\newcommand{\Cov}{\text{Cov}}
\renewcommand{\Pr}[2]{\text{Pr}_{#1} \left[ #2 \right]}

\def\RR{\mathbb R}
\def\CC{\mathbb C}
\def\QQ{\mathbb Q}
\def\ZZ{\mathbb Z}
\def\NN{\mathbb N}
\def\powset{\mathbb P}
\def\FF{\mathbb F}

\def\e{\epsilon}
\def\d{\delta}

\def\ds{\displaystyle}
\newcommand{\vs}[1]{\vspace{#1 pt}}

\def\tensor{\otimes}
\def\xor{\oplus}

\newcommand{\floor}[1]{\left\lfloor #1 \right\rfloor}
\newcommand{\ceil}[1]{\left\lceil #1 \right\rceil}
\newcommand{\field}[1]{\mathbb #1}
\newcommand{\inner}[2]{\langle #1,#2\, \rangle}
\newcommand{\norm}[2]{\| #1 \|_{#2}}
\newcommand{\ket}[1]{| #1 \rangle}
\newcommand{\bra}[1]{\langle #1 |}
\newcommand{\dirac}[2]{\langle #1 | #2\, \rangle}

\newcommand{\bracket}[1]{\langle #1 \rangle}
\newcommand{\paren}[1]{\left( #1 \right)}
\newcommand{\set}[1]{\left\{ #1 \right\}}

\newcommand{\bset}{\left\{0,1\right\}}

\newcommand{\inv}{^{-1}}
\newcommand{\til}{\widetilde}
\newcommand{\sign}{\mathrm{sgn}\;}
\renewcommand{\mod}{\text{ mod }}

\newcommand{\poly}{\text{poly}}
\newcommand{\polylog}{\text{polylog}}
\newcommand{\tsc}[1]{\textsc{#1}}

\newcommand{\Co}{{\sf Co-}}
\newcommand{\co}{{\sf co}}
\newcommand{\modpoly}{/ \text{poly}}
\newcommand{\SPACE}{{\sf SPACE}}
\newcommand{\TIME}{{\sf TIME}}
\def\D{{\sf D}}
\def\N{{\sf N}}
\def\P{{\sf P}}
\def\L{{\sf L}}
\def\E{{\sf E}}
\newcommand{\promise}{\textsf{promise}}

\newcommand{\NP}{{\sf NP}}
\newcommand{\PSPACE}{{\sf PSPACE}}
\newcommand{\EXP}{{\sf EXP}}

\newcommand{\BP}{{\sf BP}}

\newcommand{\NL}{{\sf NL}}

\newcommand{\NC}{{\sf NC}}
\newcommand{\AC}{{\sf AC}}
\newcommand{\RP}{{\sf RP}}
\newcommand{\BPP}{{\sf BPP}}
\newcommand{\PH}{{\sf PH}}
\newcommand{\PP}{{\sf PP}}
\newcommand{\IP}{{\sf IP}}
\newcommand{\AM}{{\sf AM}}
\newcommand{\MA}{{\sf MA}}
\newcommand{\PCP}{{\sf PCP}}

\newtheorem{theorem}{Theorem}[section]
\newtheorem{lemma}[theorem]{Lemma}
\newtheorem{proposition}[theorem]{Proposition}
\newtheorem{prop}[theorem]{Proposition}
\newtheorem{corollary}[theorem]{Corollary}
\newtheorem{conjecture}[theorem]{Conjecture}
%\theoremstyle{remark}

\newtheorem{remark}[theorem]{Remark}

\theoremstyle{definition}
\newtheorem{example}[theorem]{Example}
\newtheorem{problem}[theorem]{Problem}
\newtheorem{definition}[theorem]{Definition}
\newtheorem{question}[theorem]{Question}

\numberwithin{equation}{section}
\renewcommand{\theequation}{\thesection.\arabic{equation}}



\newcommand{\bp}{\textsf{BP}}
\newcommand{\strongbp}{\textsf{strongBP}}
\newcommand{\parity}{\oplus}
\newcommand{\p}{\textsf{P}}
\newcommand{\op}{\textsf{Op}}
\newcommand{\pp}{\textsf{PP}}

\begin{document}

\begin{center} \begin{LARGE} {\sc \bf Amplification with Operators on Complexity Classes} \vspace{6pt}

{\sc 18.405 Final Paper, Spring 2011} \vspace{9pt}

\end{LARGE} { \Large \textsc{Brian Hamrick and Travis Hance}}

\end{center}

\section{Introduction}

\section{Definitions}

\section{Essential Properties}

\section{Amplification of $\bp$ to \strongbp}

The aim of this section is to prove the following amplification result:

\begin{theorem}\label{amplify}
Let $C$ be a \emph{$\{\Sigma,\Pi,\bp,\parity\}$}-constructible complexity class. Then \emph{$\bp \cdot C = \strongbp \cdot C$}.
\end{theorem}
A proof of this theorem is immediate by induction from the following two lemmas:

\begin{lemma}\label{majorityimpliesamplify}
If $C$ is a complexity class which is closed under majority reductions, then\linebreak \emph{$\bp \cdot C = \strongbp \cdot C$.}
\end{lemma}

\begin{lemma}\label{ampliftymainlemma}
\begin{enumerate}
\item[] 
\item[(a)] \emph{$\p$} is closed under majority reductions.
\item[(b)] Let $C$ be a \emph{$\{\Sigma,\Pi,\bp,\parity\}$}-constructible complexity class. If $C$ is closed under majority reductions, then (i) \emph{$\Sigma \cdot C$}, (ii) \emph{$\Pi \cdot C$}, (iii) \emph{$\bp \cdot C$}, and (iv) \emph{$\parity \cdot C$} are all closed under majority reductions.
\end{enumerate}
\end{lemma}
\begin{proof}
Part (a) is trivial. For part (b), we consider (i), (ii), (iii), and (iv) separately. Part (iv) is the most difficult, and will not be finished until section \ref{oracle}.
\end{proof}

\section{An Oracle Result}\label{oracle}

To complete the proof of Lemma \ref{ampliftymainlemma} (b-iv), we use the following generalization of a result in \cite{Toda}, which shows that $\parity \cdot (\p ^{\parity\cdot\p}) = \parity\cdot\p$, a fact which is used to show that $\parity \cdot \p$ is closed under majority reductions.

\begin{theorem}\label{oracle}
Let $C$ be a class containing \emph{$\p$} which is closed under intersection. Then\linebreak \emph{$\parity \cdot (\p^{\parity \cdot C}) = \parity \cdot C$}.
\end{theorem}
\begin{proof}
The direction $\parity \cdot C \subseteq \parity \cdot (\p^{\parity \cdot C})$ is trivial. Let us consider the interesting direction\linebreak $\parity \cdot (\p^{\parity \cdot C})\subseteq \parity\cdot C$.
\end{proof}
\begin{remark}\emph{
While this paper applies Theorem \ref{oracle} only in the case where $C$ is $\{\Sigma,\Pi,\bp,\parity\}$-constructible, it can also be used to show, for instance, that $\parity \cdot (\p^{\parity \cdot \pp}) = \parity \cdot \pp$, although it is nontrivial to show that $\pp$ is closed under intersection. For a proof, see \cite{Beigel}.
}\end{remark}

\section{Conclusion}

\pagebreak

\begin{thebibliography}{9}

\bibitem{Beigel} Beigel, R., Reingold, N., and Spielman, D.A. \emph{PP is closed under intersection}, Proceedings of ACM Symposium on Theory of Computing 1991, pp. 1-9, 1991.

\bibitem{Toda}Toda, S. \emph{PP is as Hard as the Polynomial-Time Hierarchy.} Siam Journal of Computing. Vol. 20, No. 5, pp. 865-877. Oct 1991.

\bibitem{Toda2} Toda, S. and Ogiwara, M. \emph{Counting classes are at least as hard as the polynomial-time hierarchy,} Structure in Complexity Theory Conference, 1991, Proceedings of the Sixth Annual, pp. 2-12, 30 Jun-3 Jul 1991.

\end{thebibliography}

\end{document}

