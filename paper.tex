
\documentclass[11pt]{article}

\usepackage[latin1]{inputenc}
\usepackage{amssymb}
\usepackage{amsmath}
\usepackage{amscd}
\usepackage{amsthm}
\usepackage{amsfonts}
\usepackage{enumerate}
\usepackage{graphicx}
\usepackage{url}
\usepackage[breaklinks=true,hyperref]{hyperref}
\usepackage{amssymb}
\usepackage[dvips]{color}
\usepackage{epsfig}
\usepackage{mathrsfs}
\usepackage{indentfirst}
\usepackage{subfig}


\pdfpagewidth 8.5in
\pdfpageheight 11in
\topmargin -1in
\headheight 0in
\headsep 0in
\textheight 8.5in
\textwidth 6.5in
\oddsidemargin 0in
\evensidemargin 0in
\headheight 75pt
\headsep 0in
\footskip .75in


\newenvironment{ee}{\begin{enumerate}}{\end{enumerate}}
\newenvironment{ii}{\begin{itemize}}{\end{itemize}}


\newcommand{\argmax}{\arg\!\max}
\newcommand{\argmin}{\arg\!\min}

\newcommand{\Var}{\text{Var}}
\newcommand{\Cov}{\text{Cov}}
\renewcommand{\Pr}[2]{\text{Pr}_{#1} \left[ #2 \right]}

\def\RR{\mathbb R}
\def\CC{\mathbb C}
\def\QQ{\mathbb Q}
\def\ZZ{\mathbb Z}
\def\NN{\mathbb N}
\def\powset{\mathbb P}
\def\FF{\mathbb F}

\def\e{\epsilon}
\def\d{\delta}

\def\ds{\displaystyle}
\newcommand{\vs}[1]{\vspace{#1 pt}}

\def\tensor{\otimes}
\def\xor{\oplus}

\newcommand{\floor}[1]{\left\lfloor #1 \right\rfloor}
\newcommand{\ceil}[1]{\left\lceil #1 \right\rceil}
\newcommand{\field}[1]{\mathbb #1}
\newcommand{\inner}[2]{\langle #1,#2\, \rangle}
\newcommand{\norm}[2]{\| #1 \|_{#2}}
\newcommand{\ket}[1]{| #1 \rangle}
\newcommand{\bra}[1]{\langle #1 |}
\newcommand{\dirac}[2]{\langle #1 | #2\, \rangle}

\newcommand{\bracket}[1]{\langle #1 \rangle}
\newcommand{\paren}[1]{\left( #1 \right)}
\newcommand{\set}[1]{\left\{ #1 \right\}}

\newcommand{\bset}{\left\{0,1\right\}}

\newcommand{\inv}{^{-1}}
\newcommand{\til}{\widetilde}
\newcommand{\sign}{\mathrm{sgn}\;}
\renewcommand{\mod}{\text{ mod }}

\newcommand{\poly}{\text{poly}}
\newcommand{\polylog}{\text{polylog}}
\newcommand{\tsc}[1]{\textsc{#1}}

\newcommand{\Co}{{\sf Co-}}
\newcommand{\co}{{\sf co}}
\newcommand{\modpoly}{/ \text{poly}}
\newcommand{\SPACE}{{\sf SPACE}}
\newcommand{\TIME}{{\sf TIME}}
\def\D{{\sf D}}
\def\N{{\sf N}}
\def\P{{\sf P}}
\def\L{{\sf L}}
\def\E{{\sf E}}
\newcommand{\promise}{\textsf{promise}}

\newcommand{\NP}{{\sf NP}}
\newcommand{\PSPACE}{{\sf PSPACE}}
\newcommand{\EXP}{{\sf EXP}}

\newcommand{\BP}{{\sf BP}}

\newcommand{\NL}{{\sf NL}}

\newcommand{\NC}{{\sf NC}}
\newcommand{\AC}{{\sf AC}}
\newcommand{\RP}{{\sf RP}}
\newcommand{\BPP}{{\sf BPP}}
\newcommand{\PH}{{\sf PH}}
\newcommand{\PP}{{\sf PP}}
\newcommand{\IP}{{\sf IP}}
\newcommand{\AM}{{\sf AM}}
\newcommand{\MA}{{\sf MA}}
\newcommand{\PCP}{{\sf PCP}}

\newtheorem{theorem}{Theorem}[section]
\newtheorem{lemma}[theorem]{Lemma}
\newtheorem{proposition}[theorem]{Proposition}
\newtheorem{prop}[theorem]{Proposition}
\newtheorem{corollary}[theorem]{Corollary}
\newtheorem{conjecture}[theorem]{Conjecture}
%\theoremstyle{remark}

\newtheorem{remark}[theorem]{Remark}

\theoremstyle{definition}
\newtheorem{example}[theorem]{Example}
\newtheorem{problem}[theorem]{Problem}
\newtheorem{definition}[theorem]{Definition}
\newtheorem{question}[theorem]{Question}

\numberwithin{equation}{section}
\renewcommand{\theequation}{\thesection.\arabic{equation}}



\newcommand{\pr}{\text{Pr}}

\newcommand{\bp}{\textsf{BP}}
\newcommand{\strongbp}{\widehat{\textsf{BP}}}
%\newcommand{\strongbp}{\textsf{strongBP}}
\newcommand{\bpp}{\textsf{BPP}}
\newcommand{\parity}{\oplus}
\newcommand{\p}{\textsf{P}}
\newcommand{\op}{\textsf{Op}}
\newcommand{\pp}{\textsf{PP}}
\newcommand{\x}{\textsf{X}}
\newcommand{\intersection}{\textsf{intersect}}
\newcommand{\intersect}{\textsf{intersect}}
\newcommand{\majority}{\textsf{majority}}

\begin{document}

\begin{center} \begin{LARGE} {\sc \bf Amplification with Operators on Complexity Classes} \vspace{6pt}

{\sc 18.405 Final Paper, Spring 2011} \vspace{9pt}

\end{LARGE} { \Large \textsc{Brian Hamrick and Travis Hance}}

\end{center}

\section{Introduction}

\section{Definitions}

\subsection{Partial Functions}

Complexity classes are usually defined in terms of \emph{languages}. A language $L$ is a subset of $\bigcup_{n=0}^{\infty}\{0,1\}^n$, which can also be viewed as a function $f_L: \bigcup_{n=0}^{\infty}\{0,1\}^n \to \{0,1\}$, where $f_L(x) = 1$ if and only if $x \in L$. However, the definitions of some complexity classes in terms of languages leads to undesirable consequences. For example, the standard definition of $\bpp$ means that a ``$\bpp$-machine'' must accept \emph{every} input with probability outside the range $[\frac{1}{3},\frac{2}{3}]$, but nearly every natural problem whose solution is described as a $\bpp$ algorithm in fact has a promise associated. The concept of a partial function is a natural way to resolve this issue.

\begin{definition}\label{partialfunction}
A \emph{partial function} is a function $f: \bigcup_{n=0}^{\infty}\{0,1\}^n \to \{0,1,\x\}$.
\end{definition}

\subsection{Complexity Classes}

For the entirety of this paper, we will use the following definitions of a complexity class and $\p$.

\begin{definition}\label{complexityclass}
A \emph{complexity class} is a set of partial functions.
\end{definition}

\begin{definition}\label{p}
The complexity class $\p$ is the set of partial functions such that $f \in \p$ if and only if there exists a polynomial $p$ and a Turing machine $T$ such that:
\begin{itemize}
\item If $f(x) = 1$, then $T$ halts after at most $p(|x|)$ steps and accepts.
\item If $f(x) = 0$, then $T$ halts after at most $p(|x|)$ steps and rejects.
\end{itemize}
\end{definition}

Notice that there is no restriction on $T$ if $f(x) = \x$. However, this definition of $\p$ can be reconciled with the standard definition of $\p$ in the following way.

\begin{remark}
For every partial function $f \in \p$, there exists another partial function $L \in \p$ such that
\begin{itemize}
\item For all $x$, $L(x) = 0$ or $L(x) = 1$.
\item If $f(x) = 1$, then $L(x) = 1$.
\item If $f(x) = 0$, then $L(x) = 0$.
\end{itemize}
\end{remark}

To see this, consider simulating $T$ and cutting it off after $p(|x|)$ steps, which is followed by a rejection. In this statement, we can see $L$ as a language in the standard complexity class $\p$, so effectively any partial function can be extended to a language.

\subsection{Operators}

%\section{Closed Under Intersection}
\section{Amplification of $\bp$ to $\strongbp$}

Consider the amplification of the probabilities in the $\bp$ operator. It is well known that\linebreak $\bpp = \bp \cdot \p$ can be amplified; that is, we can replace the error probability of $1/3$ with probabilities exponentially small in the input size. Thus we can write $\bp \cdot \p = \strongbp \cdot \p$. We would like to compare $\bp \cdot C$ to $\strongbp \cdot C$ for more general complexity classes $C$.

The aim of this section is to prove the following amplification result:

\begin{theorem}\label{amplify}
Let $C$ be a \emph{$\{\Sigma,\Pi,\bp,\parity\}$}-constructible complexity class. Then \emph{$\bp \cdot C = \strongbp \cdot C$}.
\end{theorem}
Initially, this may seem to be obviously true for the same reasons that $\bp \cdot \p = \strongbp \cdot \bp$. The problem is that amplification requires running some machine which accepts a language in $C$ more than once. This requires the structure of $C$ to allow for such repetitions. This brings us to the concept of majority reductions.
\begin{definition}\label{defmajority}\emph{
We say that a complexity class $C$ is \emph{closed under majority reductions} if for any language $L$, the language $\majority(L) = \{(x_1,...,x_k) : k \in \mathbb{N}, |\{x_i : x_i \in L\}| \ge k/2\}$ is in $C$.
}\end{definition}
Now, we see that majority reductions are exactly what we want for amplification:
\begin{lemma}\label{majorityimpliesamplify}
If $C$ is a complexity class which is closed under majority reductions, then\linebreak \emph{$\bp \cdot C = \strongbp \cdot C$.}
\end{lemma}
\begin{proof} \end{proof}
Thus, in order to prove Theorem \ref{amplify}, we just need to show that any \emph{$\{\Sigma,\Pi,\bp,\parity\}$}-constructible complexity class is closed under majority reductions. In order to show that these are closed under majority reductions, we will need the help of another property:
\begin{definition}\label{defintersection}\emph{
We say that a class $C$ is \emph{closed under intersection} if for any language $L \in C$, the language $\intersection(L) = \{(x_1,...,x_k) : k\in\mathbb{N}, x_i \in L_i ~ \forall i\}$ is in $C$.
}\end{definition}
We will demonstrate that all $\{\Sigma, \Pi, \bp, \parity\}$-constructible complexity classes are closed under intersection and under majority reductions. To do this, we will use induction on the number of operators in $C$. Clearly $\p$ is closed under intersection and majority reductions. Thus, the result follows from the following two lemmas:

\begin{lemma}\label{intersectionlemma}
Let $C$ be a \emph{$\{\Sigma,\Pi,\bp,\parity\}$}-constructible complexity class. If $C$ is closed under majority reductions and under intersection, then (i) \emph{$\Sigma \cdot C$}, (ii) \emph{$\Pi \cdot C$}, (iii) \emph{$\bp \cdot C$}, and (iv) \emph{$\parity \cdot C$} are all closed under intersection.
%Let $C$ be a \emph{$\{\Sigma,\Pi,\bp,\parity\}$}-constructible complexity class. Then $C$ is closed under intersection.
\end{lemma}

\begin{lemma}\label{amplifymainlemma}
Let $C$ be a \emph{$\{\Sigma,\Pi,\bp,\parity\}$}-constructible complexity class. If $C$ is closed under majority reductions, then (i) \emph{$\Sigma \cdot C$}, (ii) \emph{$\Pi \cdot C$}, (iii) \emph{$\bp \cdot C$}, and (iv) \emph{$\parity \cdot C$} are all closed under majority reductions.
\end{lemma}
We begin with the proof of Lemma \ref{intersectionlemma}
\begin{proof} \emph{(Lemma \ref{intersectionlemma})}

Suppose $L \in \Sigma\cdot C$. Then there exists a $C$ machine $M$ such that $x \in L$ if and only if there exists some $y$ (of length polynomial in the length of $x$) such that $M$ accepts $(x, y)$. Now, we want to construct a $\Sigma \cdot C$ machine for $\intersect(L)$; that is, given a $k$-tuple $(x_1,x_2,...,x_k)$, we want to determine if all $x_i$ are in $L$. To do this, we just guess a $k$-tuple $(y_1,y_2,...,y_k)$, and then check if $M$ accepts $(x_1, y_1)$, $(x_2, y_2)$, ..., and $(x_k, y_k)$. This is equivalent to checking that $((x_1,y_1),(x_2,y_2),...,(x_k,y_k)) \in \intersect(M)$, and by assumption, $\intersect(M)$ is in $C$ since $M$ is in $C$.

The proof for $\Pi \cdot C$ is analogous to the proof for $\Sigma \cdot C$.

Now we consider $\bp \cdot C$. By assumption, $C$ is closed under majority reductions, so by Lemma \ref{majorityimpliesamplify}, $\bp \cdot C = \strongbp \cdot C$. If we take any language $L \in \bp \cdot C$, we find that $L \in \strongbp \cdot C$. Therefore, there exists a $C$ machine $M$ such that $x\in L$ if and only if
\begin{center}
$\displaystyle \pr_y [M(x,y)\text{ accepts}] > $
\end{center}
\end{proof}

\begin{proof} \emph{(Lemma \ref{amplifymainlemma})}

We consider (i), (ii), (iii), and (iv) separately. Part (iv) is the most difficult, and will not be finished until Section \ref{oracle}.
\end{proof}

\section{An Oracle Result}\label{oracle}

To complete the proof of Lemma \ref{amplifymainlemma} (b-iv), we use the following generalization of a result in \cite{Toda}, which shows that $\parity \cdot (\p ^{\parity\cdot\p}) = \parity\cdot\p$, a fact which is used to show that $\parity \cdot \p$ is closed under majority reductions.

\begin{theorem}\label{oracleparityc}
Let $C$ be a class containing \emph{$\p$} which is closed under intersection. Then\linebreak \emph{$\parity \cdot (\p^{\parity \cdot C}) = \parity \cdot C$}.
\end{theorem}
\begin{proof}
The direction $\parity \cdot C \subseteq \parity \cdot (\p^{\parity \cdot C})$ is trivial. Let us consider the interesting direction\linebreak $\parity \cdot (\p^{\parity \cdot C})\subseteq \parity\cdot C$.
\end{proof}
\begin{remark}\emph{
While this paper applies Theorem \ref{oracle} only in the case where $C$ is $\{\Sigma,\Pi,\bp,\parity\}$-constructible, it can also be used to show, for instance, that $\parity \cdot (\p^{\parity \cdot \pp}) = \parity \cdot \pp$, although it is nontrivial to show that $\pp$ is closed under intersection. For a proof, see \cite{Beigel}.
}\end{remark}

\section{Conclusion}

\pagebreak

\begin{thebibliography}{9}

\bibitem{Beigel} Beigel, R., Reingold, N., and Spielman, D.A. \emph{PP is closed under intersection}, Proceedings of ACM Symposium on Theory of Computing 1991, pp. 1-9, 1991.

\bibitem{Toda}Toda, S. \emph{PP is as Hard as the Polynomial-Time Hierarchy.} Siam Journal of Computing. Vol. 20, No. 5, pp. 865-877. Oct 1991.

\bibitem{Toda2} Toda, S. and Ogiwara, M. \emph{Counting classes are at least as hard as the polynomial-time hierarchy,} Structure in Complexity Theory Conference, 1991, Proceedings of the Sixth Annual, pp. 2-12, 30 Jun-3 Jul 1991.

\end{thebibliography}

\end{document}

