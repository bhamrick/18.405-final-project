
\documentclass[11pt]{article}

\usepackage[latin1]{inputenc}
\usepackage{amssymb}
\usepackage{amsmath}
\usepackage{amscd}
\usepackage{amsthm}
\usepackage{amsfonts}
\usepackage{enumerate}
\usepackage{graphicx}
\usepackage{url}
\usepackage[breaklinks=true,hyperref]{hyperref}
\usepackage{amssymb}
\usepackage[dvips]{color}
\usepackage{epsfig}
\usepackage{mathrsfs}
\usepackage{indentfirst}
\usepackage{subfig}

\include{header}

\newcommand{\bp}{\textsf{BP}}
\newcommand{\strongbp}{\textsf{strongBP}}
\newcommand{\parity}{\oplus}
\newcommand{\p}{\textsf{P}}
\newcommand{\op}{\textsf{Op}}

\begin{document}

\begin{center} \begin{LARGE} {\sc \bf Amplification with Operators on Complexity Classes} \vspace{6pt}

{\sc 18.405 Final Paper, Spring 2011} \vspace{9pt}

\end{LARGE} { \Large \textsc{Brian Hamrick and Travis Hance}}

\end{center}

\section{Introduction}

\section{Definitions}

\section{Essential Properties}

\section{Amplification of $\bp$ to \strongbp}

The aim of this section is to prove the following amplification result:

\begin{theorem}\label{amplify}
Let $C$ be a \emph{$\{\Sigma,\Pi,\bp,\parity\}$}-constructible complexity class. Then \emph{$\bp \cdot C = \strongbp \cdot C$}.
\end{theorem}
A proof of this theorem is immediate by induction from the following two lemmas:

\begin{lemma}\label{majorityimpliesamplify}
If $C$ is a complexity class which is closed under majority reductions, then\linebreak \emph{$\bp \cdot C = \strongbp \cdot C$.}
\end{lemma}

\begin{lemma}\label{ampliftymainlemma}
\begin{enumerate}
\item[] 
\item[(a)] \emph{$\p$} is closed under majority reductions.
\item[(b)] Let $C$ be a \emph{$\{\Sigma,\Pi,\bp,\parity\}$}-constructible complexity class. If $C$ is closed under majority reductions, then (i) \emph{$\Sigma \cdot C$}, (ii) \emph{$\Pi \cdot C$}, (iii) \emph{$\bp \cdot C$}, and (iv) \emph{$\parity \cdot C$} are all closed under majority reductions.
\end{enumerate}
\end{lemma}
\begin{proof}
Part (a) is trivial. For part (b), we consider (i), (ii), (iii), and (iv) separately. Part (iv) is the most difficult, and will not be finished until section \ref{oracle}.
\end{proof}

\section{An Oracle Result}\label{oracle}

To complete the proof of Lemma \ref{ampliftymainlemma} (b-iv), we use the following generalization of a result in \cite{Toda}, which shows that $\parity \cdot (\p ^{\parity\cdot\p}) = \parity\cdot\p$, a fact which is used to show that $\parity \cdot \p$ is closed under majority reductions.

\begin{theorem}\label{oracle}
Let $C$ be a class containing \emph{$\p$} which is closed under intersection. Then\linebreak \emph{$\parity \cdot (\p^{\parity \cdot C}) = \parity \cdot C$}.
\end{theorem}
\begin{proof}

\end{proof}

\section{Conclusion}

\pagebreak

\begin{thebibliography}{9}

\bibitem{Toda}Toda, S. \emph{PP is as Hard as the Polynomial-Time Hierarchy.} Siam Journal of Computing. Vol. 20, No. 5, pp. 865-877. Oct 1991.

\bibitem{Toda2} Toda, S.; Ogiwara, M. \emph{Counting classes are at least as hard as the polynomial-time hierarchy,} Structure in Complexity Theory Conference, 1991, Proceedings of the Sixth Annual, pp. 2-12, 30 Jun-3 Jul 1991.

\end{thebibliography}

\end{document}

